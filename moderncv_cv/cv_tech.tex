\documentclass[10pt,a4paper]{moderncv}
\moderncvtheme[blue]{classic}                
\usepackage[utf8]{inputenc}
%\usepackage[top=1.1cm, bottom=1.1cm, left=2cm, right=2cm]{geometry}
\usepackage[top=0.9cm, bottom=0.6cm, left=2cm, right=2cm]{geometry}
\setlength{\hintscolumnwidth}{2.8cm}

\firstname{Andrei}
\familyname{Chaplygin}
\title{Software developer} 
\address{Moscow, Russia}{}
\mobile{+79651818175} 
\email{achaplygin99@gmail.com} 

\begin{document}
\maketitle
%\vspace*{-1.5\baselineskip}

\section{Work Experience}

\cventry{2019-present}{C++ Senior software developer}{{Rock Flow Dynamics}}{}{}
{tNavigator - high-performance tool for integrated static and dynamic modelling from reservoir to surface networks:
  \begin{itemize}
  \item Worked with a team of 15 people, participated in code reviewing and team-leading
  \item Developed numerical methods for fully implicit reservoir simulation and surface network modeling
  \item Implemented various graph algorithms for surface network modeling
  \item Participated in development of solver for Sequential Quadratic Programming optimization problem
  \item Worked with large code base, optimized critical code sections 
  %\item Developed graph algorithms for surface networks
  \end{itemize}
}

\cventry{2016-2019}{R\&D}{{Schlumberger}}{}{}
{
\begin{itemize}
 \item Simulated multiphase flow using OpenFoam and fast extension hydraulic fractures
 \item Developed the simulator FastRadialHF for the initiation and propagation hydraulic fractures
 \item Implemented an efficient nonlinear solver for the fully coupled multiphysics matrix
 \item Wrote article with research results in Q1 scientific journal 
\end{itemize}
}

\cventry{2015-2016}{C++ software developer}{{Geometric Modeling and Interactive Systems Research Group}}{}{}
{
\begin{itemize}
 \item Developed the software for visualization of satellites navigation
 \item Implemented numerical methods of orbital dynamics and computer graphic algorithms
 \item Worked with Qt, OpenSceneGraph, OpenGL shaders
\end{itemize}
}

\section{Education}

\cventry{2019-present}{PhD in Applied Mathematics and Computer Science}{Marchuk Institute of Numerical Mathematics}{}{}
{Thesis: Improvement of the general ocean circulation model for efficient use on massively parallel and heterogeneous computing systems}

\cventry{2017-2019}{Master degree in Applied Mathematics and Computer Science}{Lomonosov Moscow State University}{}{}
{Thesis: Load balancing method using Hilbert space-filling curves for INMOM
(Institute of Numerical Mathematics Ocean Model)}

\cventry{2013-2017}{Bachelor degree in Applied Mathematics and Computer Science}{Lomonosov Moscow State University}{}{}
{Thesis: Implementation of parallel INMOM (Institute of Numerical Mathematics Ocean Model) ocean circulation model}

\section{Research Experience}

\cventry{2015-present}{Multi-scale mathematical modeling of the atmosphere and ocean}{}{}{}
{INMOM - general ocean circulation model which has been used as the oceanic block of the climate model INMCM. This coupled model is representative at various stages of the international project for comparing climate models CMIP, conducted under the auspices of IPCC.
  \begin{itemize}
  \item Developed the software architecture separating the physics-related code from features of parallel implementation
  \item Implemented various hybrid parallel programming patterns using MPI, OpenMP, CUDA
  \item Developed parallel algorithms for the ocean model 
  \item Evaluated scaling performances on modern supercomputers
  \item Literature review of climate change problems
  \end{itemize}
}

\section{Programming Skills}
\cvcomputer{Programming Languages}{C++ (98/11/17), C, Fortran, Python, Wolfram Mathematica}{Parallel Computing}{Threads, MPI, OpenMP, CUDA, Slurm, TAU Performance System}
\cvcomputer{Development tools}{git, bash, gdb, ssh, sanitizers (asan, tsan), valgring, Qt Creator, Visual Studio, vim, ect.}{Libraries}{STL, Qt, PETSc, Intel MKL (Lapack, BLAS), GSL, OpenFoam, OpenGL, OpenSceneGraph, etc.}
\cvcomputer{Build systems}{CMake, Ninja, Jenkins}{General}{\LaTeX, OOP, test driven development, agile methodology, etc.}

\clearpage

\section{Languages}
\cvlanguage{Russian}{Fluent}{}
\cvlanguage{English}{Upper-Intermediate}{}

\section{Publications}

\cvline{}
{Chaplygin, A. V., Gusev, A. V., Diansky, N. A. High-performance Shallow Water Model for Use on Massively Parallel and Heterogeneous Computing Systems. Supercomputing Frontiers and Innovations, 8(4), 2022}

\cvline{}
{Fomin, V. V., Panasenkova, I. I., Gusev A. V., Chaplygin, A. V., Diansky, N. A. Operational forecasting system for Arctic Ocean using the Russian marine circulation model INMOM-Arctic. Arctic: Ecology and Economy, vol. 11, no. 2, 2021}

\cvline{}
{Chaplygin, A.V., Gusev, A.V. Shallow Water Model Using a Hybrid MPI/OpenMP Parallel Programming. Problems of Informatics 1, 2021}

\cvline{}
{Maxim Chertov, Andrey Chaplygin. Evaluating characteristics of high-rate hydraulic fractures driven by wellbore energy source. Engineering Fracture Mechanics, Volume 222, 2019}

\cvline{}
{Chaplygin A.V., Diansky N.A., and Gusev A.V. Load balancing using Hilbert space-filling curves for parallel shallow water simulations. Numerical methods and programming. Vol. 20, 2019}

\cvline{}
{Diansky N.A, Fomin V.V., Grigoriev A.V., Chaplygin A.V., Zatsepin A.G. Spatial-Temporal Variability of Inertial Currents in the Eastern Part of the Black Sea in a Storm Period. Physical Oceanography. Vol. 26, Iss. 2, 2019}

\section{Presentations}

\cvline{}{Shallow water model using a hybrid MPI/OpenMP parallel programming. Mathematical modeling and supercomputer technologies, Nizhny Novgorod, Russia, 2020}

\cvline{}{A full free surface ocean general circulation model in sigma-coordinates for simulation of the World Ocean circulation and its variability. EGU General Assembly, Vienna, Austria, 2019} 

\cvline{}{Parallel modeling of nonlinear shallow water equations. 60th MIPT Scientific Conference, Moscow, Russia, 2017}

\cvline{}{Calculation of extreme surge in the Taganrog Bay and the use of atmospheric and ocean circulation models of different spatial resolution. International Scientific Conference Marine Research and Education, Moscow, Russia, 2017}

\section{Activities}
\cvline{}{Rome-Moscow school of Matrix Methods and Applied Linear Algebra 2018 (participation)}
\cvline{}{Rome-Moscow school of Matrix Methods and Applied Linear Algebra 2016 (participation)}

\section{Links}
\cvline{}{\httplink{https://github.com/Andrcraft9}}
\cvline{}{\httplink{https://www.linkedin.com/in/andrey-chaplygin-1917a4193}}
\cvline{}{\httplink{https://www.researchgate.net/profile/Andrey-Chaplygin}}

\end{document}


