\documentclass[paper=a4,fontsize=11pt]{scrartcl}
\usepackage[english]{babel}
\usepackage[utf8x]{inputenc}
\usepackage[protrusion=true,expansion=true]{microtype}
\usepackage{amsmath,amsfonts,amsthm}     % Math packages
\usepackage{graphicx}                    % Enable pdflatex
\usepackage[svgnames]{xcolor}            % Colors by their 'svgnames'
\usepackage{geometry}
\textheight=700px                        % Saving trees ;-)
\usepackage{url}

\usepackage{indentfirst}

\frenchspacing              % Better looking spacings after periods
\pagestyle{empty}           % No pagenumbers/headers/footers

%%% Custom sectioning (sectsty package)
%%% ------------------------------------------------------------
\usepackage{sectsty}

\sectionfont{%	                   % Change font of \section command
\usefont{OT1}{phv}{b}{n}%	   % bch-b-n: CharterBT-Bold font
\sectionrule{0pt}{0pt}{-5pt}{3pt}}

%%% Macros
%%% ------------------------------------------------------------

\DeclareOldFontCommand{\bf}{\normalfont\bfseries}{\mathbf}

\newlength{\spacebox}
\settowidth{\spacebox}{8888888888}		% Box to align text
\newcommand{\sepspace}{\vspace*{1em}}		% Vertical space macro

\newcommand{\MyName}[1]{ % Name
		\Huge \usefont{OT1}{phv}{b}{n} \hfill #1
		\par \normalsize \normalfont}
		
\newcommand{\MySlogan}[1]{ % Slogan (optional)
		\large \usefont{OT1}{phv}{m}{n}\hfill \textit{#1}
		\par \normalsize \normalfont}

\newcommand{\NewPart}[1]{\section*{\uppercase{#1}}}

\newcommand{\PersonalEntry}[2]{
		\noindent\hangindent=2em\hangafter=0 % Indentation
		\parbox{\spacebox}{        % Box to align text
		\textit{#1}}		       % Entry name (birth, address, etc.)
		\hspace{1.5em} #2 \par}    % Entry value

\newcommand{\SkillsEntry}[2]{      % Same as \PersonalEntry
		\noindent\hangindent=2em\hangafter=0 % Indentation
		\parbox{\spacebox}{        % Box to align text
		\textit{#1}}			   % Entry name (birth, address, etc.)
		\hspace{1.5em} #2 \par}    % Entry value	
		
\newcommand{\EducationEntry}[4]{
		\noindent \textbf{#1} \hfill      % Study
		\colorbox{Black}{%
			\parbox{6em}{%
			\hfill\color{White}#2}} \par  % Duration
		\noindent \textit{#3} \par        % School
		\noindent\hangindent=2em\hangafter=0 \small #4 % Description
		\normalsize \par}

\newcommand{\WorkEntry}[4]{				  % Same as \EducationEntry
		\noindent \textbf{#1} \hfill      % Jobname
		\colorbox{Black}{\color{White}#2} \par  % Duration
		\noindent \textit{#3} \par              % Company
		\noindent\hangindent=2em\hangafter=0 \small #4 % Description
		\normalsize \par}

%%% Begin Document
%%% ------------------------------------------------------------
\begin{document}

% you can upload a photo and include it here...
%\begin{wrapfigure}{l}{0.5\textwidth}
%	\vspace*{-2em}
%		\includegraphics[width=0.15\textwidth]{photo}
%\end{wrapfigure}

\MyName{Chaplygin Andrei}
\MySlogan{Software developer}

\sepspace

%%% Personal details
%%% ------------------------------------------------------------
\NewPart{Personal details}{}

%\PersonalEntry{Birth}{January 1, 1980}
\PersonalEntry{Address}{Moscow, Russia}
\PersonalEntry{Phone}{+79651818175}
\PersonalEntry{Email}{\url{achaplygin99@gmail.com}}
\PersonalEntry{Github}{\url{https://github.com/Andrcraft9}}
\PersonalEntry{LinkedIn}{\url{https://www.linkedin.com/in/andrey-chaplygin-1917a4193/}}

%%% Work experience
%%% ------------------------------------------------------------
\NewPart{Work experience}{}

\EducationEntry{Rock Flow Dynamics}{2019-present}{Senior C++ Developer}
{
tNavigator - high-performance tool for integrated static and dynamic modelling from reservoir to surface networks.
}

\EducationEntry{Faculty of Space Research, Lomonosov Moscow State University }{2019}{C++ Developer}
{
Developing the software for autonomous satellite navigation. Software allows to estimate maximum permissible errors for measurements of satellite devices. Working with non-linear least square optimization problem. 
}
\sepspace

\EducationEntry{Schlumberger}{2016-2019}{Researcher} %Researcher-Intern
{
\begin{itemize}
 \item
 
 %We looked at energetic driven pumping. 
 We simulated multiphase incompressible/compressible flow in cylindrical tube using OpenFoam and compared different numerical schemes and algorithms.
 
 \item We developed the simulator FastRadialHF for the initiation and propagation hydraulic fractures. 
A model of high injection rate hydraulic fractures driven by wellbore pressure pulse was developed. 
%The model describes a fully coupled wellbore-fracture system with turbulent fluid friction and compressibility.
An efficient nonlinear solver was implemented for the fully coupled multiphysics matrix.
The influence of wellbore energy source parameters on fracture characteristics was studied.
%Simulating fast extension hydraulic fractures, .
\end{itemize}

}
\sepspace

\EducationEntry{Geometric Modeling and Interactive Systems Research Group}{2015-2016}{C++ Developer}
{Developing the software for visualization of navigation satellite groups motion.
Software allows to observe the simulated work of the GLONASS system, 
where orbital satellite groups and ground stations were displayed 
using computer graphics and modern methods of mathematical modeling.}

%%% Education / Research experience
%%% ------------------------------------------------------------
\NewPart{Research experience}{}

\EducationEntry{PhD.}{2019-present}{Marchuk Institute of Numerical Mathematics}
{{\bf Thesis: Improvement of the general ocean circulation model for efficient use on massively parallel and heterogeneous computing systems.}}
\sepspace

\EducationEntry{MSc.}{2017-2019}{Lomonosov Moscow State University}
{Faculty of Computational Mathematics and Cybernetics, 
Department of Computational Technologies and Modeling.

{\bf Thesis: Load balancing method using Hilbert space-filling curves for INMOM
(Institute of Numerical Mathematics Ocean Model)}}
\sepspace

\EducationEntry{BSc.}{2013-2017}{Lomonosov Moscow State University}
{Faculty of Computational Mathematics and Cybernetics, 
Department of Computational Technologies and Modeling.

{\bf Thesis: Implementation of parallel INMOM (Institute of Numerical Mathematics Ocean Model) ocean circulation model}}

%%% Skills
%%% ------------------------------------------------------------
\NewPart{Skills}{}

%\SkillsEntry{Languages}{Russian (mother tongue)}
%\SkillsEntry{}{English (not fluent)}
%\SkillsEntry{Software}{\textsc{Matlab}, \LaTeX, \textsc{Ansys}, \textsc{Comsol}}
\SkillsEntry{Programming Languages}{\textsc{C++}, \textsc{C}, \textsc{Fortran}, \textsc{Python}, \textsc{Wolfram Mathematica}}
\sepspace

\SkillsEntry{Parallel Computing}{\textsc{MPI}, \textsc{OpenMP}, \textsc{CUDA}}
\sepspace

\SkillsEntry{Computer Graphics}{\textsc{OpenSceneGraph}, \textsc{OpenGL}}
\sepspace

\SkillsEntry{Other}{\textsc{Git}, \LaTeX, \textsc{Qt}, \textsc{PETSc}, \textsc{GSL}, \textsc{OpenFoam}, \textsc{Lapack/BLAS}}

%%% Publications
%%% ------------------------------------------------------------
%\newpage
\NewPart{Publications}{}

Chaplygin, A. V., Gusev, A. V., Diansky, N. A.

High-performance Shallow Water Model for Use on Massively Parallel and Heterogeneous Computing Systems. 
\textit{Supercomputing Frontiers and Innovations, 8(4), 2022}
\sepspace

Fomin, V. V., Panasenkova, I. I., Gusev A. V., Chaplygin, A. V., Diansky, N. A.

Operational forecasting system for Arctic Ocean using the Russian marine circulation model INMOM-Arctic.
\textit{Arctic: Ecology and Economy, vol. 11, no. 2, 2021}
\sepspace

Chaplygin, A.V., Gusev, A.V. 

Shallow Water Model Using a Hybrid MPI/OpenMP Parallel Programming. 
\textit{Problems of Informatics 1, 2021}
\sepspace

Maxim Chertov, Andrey Chaplygin.

Evaluating characteristics of high-rate hydraulic fractures driven by wellbore energy source.
\textit{Engineering Fracture Mechanics, Volume 222, 2019}
\sepspace

Chaplygin A.V., Diansky N.A., and Gusev A.V. 

Load balancing using Hilbert space-filling curves for parallel shallow water simulations.
\textit{Numerical methods and programming. Vol.20, 2019.}
\sepspace

Diansky N.A, Fomin V.V., Grigoriev A.V., Chaplygin A.V., Zatsepin A.G.

Spatial-Temporal Variability of Inertial Currents
in the Eastern Part of the Black Sea in a Storm Period.
\textit{Physical Oceanography. Vol.26, Iss.2, 2019.}
\sepspace

%%% Presentation
%%% ------------------------------------------------------------
\NewPart{Presentations}{} % Conferences

Chaplygin A.V., Diansky N.A., and Gusev A.V. 

%Параллельное моделирование нелинейных уравнений мелкой воды. 
Parallel modeling of nonlinear shallow water equations.
%Труды 60-й Всероссийской научной конференции МФТИ, 2017
\textit{60th MIPT Scientific Conference, 2017.}
\sepspace

Anatoly Gusev, Andrey Chaplygin, Nikolay Diansky. 

A full free surface ocean general circulation model in sigma-coordinates for simulation of the World Ocean circulation and its variability. 
\textit{EGU General Assembly 2019} %Geophysical Research Abstracts Vol. 21 
\sepspace

Fomin V.V., Diansky N.A., Chaplygin A.V.

Calculation of extreme surge in the Taganrog Bay and the use of atmospheric and ocean circulation models of different spatial resolution.
%Расчет экстремальных нагонов в Таганрогском заливе и использование моделей циркуляции атмосферы и океана различного пространственного разрешения.
\textit{International Scientific Conference Marine Research and Education 2017. }

%%% Activities
%%% ------------------------------------------------------------
\NewPart{Activities}{}
\begin{itemize}
 \item
Rome-Moscow school of Matrix Methods and Applied Linear Algebra 2018. 
\sepspace

\item
Rome-Moscow school of Matrix Methods and Applied Linear Algebra 2016.

\end{itemize}

\end{document}